
\documentclass[10pt,a4paper,twoside]{article}
\usepackage[dutch]{babel}
\usepackage{amssymb}
\usepackage{amsmath}
\usepackage{float,flafter}	
\usepackage{hyperref}
\usepackage{inputenc}
\setlength\paperwidth{20.999cm}\setlength\paperheight{29.699cm}\setlength\voffset{-1in}\setlength\hoffset{-1in}\setlength\topmargin{1cm}\setlength\headheight{12pt}\setlength\headsep{0cm}\setlength\footskip{1.131cm}\setlength\textheight{25cm}\setlength\oddsidemargin{2.499cm}\setlength\textwidth{15.999cm}

\begin{document}
\begin{center}
\hrule

\vspace{.4cm}
{\bf {\Large ASSIGNMENT-2 }}\\
\vspace{.3cm}
{\bf {\huge  Godel’s incompleteness theorems  }}
\vspace{.3cm}
\end{center}
{\bf Name:}  Shivam Yadav\\
{\bf Roll no:}  19111054 \\
{\bf Branch: }  Biomedical Engineering \hspace{\fill}  22 July, 2021 \\
\hrule

\vspace{.5cm}






\section{Introduction}
Gödel's incompleteness theorems are two theorems of mathematical logic that are concerned with the limits of provability in formal axiomatic theories. The theorems are widely, but not universally, interpreted as showing that Hilbert's program to find a complete and consistent set of axioms for all mathematics is impossible.
Gödel's incompleteness theorems are two theorems of mathematical logic that are concerned with the limits of provability in formal axiomatic theories
 

\subsection{First Incompleteness Theorem  }
”Any consistent formal system F within which a certain amount of elementary arithmetic
can be carried out is incomplete; i.e., there are statements of the language of F which can neither be proved nor disproved in F.”\\
Gödel's first incompleteness theorem first appeared as "Theorem VI" in Gödel's 1931 paper "On Formally Undecidable Propositions of Principia Mathematica and Related Systems I". The hypotheses of the theorem were improved shortly thereafter by J. Barkley Rosser (1936) using Rosser's trick. The resulting theorem (incorporating Rosser's improvement) may be paraphrased in English as follows, where "formal system" includes the assumption that the system is effectively generated.

First Incompleteness Theorem: "Any consistent formal system F within which a certain amount of elementary arithmetic can be carried out is incomplete; i.e., there are statements of the language of F which can neither be proved nor disproved in F."

\subsection{Second Incompleteness theorem}
The second incompleteness theorem, an extension of the first, shows that the system cannot demonstrate its own consistency. Employing a diagonal argument, Gödel's incompleteness theorems were the first of several closely related theorems on the limitations of formal systems.\\

For each formal system F containing basic arithmetic, it is possible to canonically define a formula Cons(F) expressing the consistency of F. This formula expresses the property that "there does not exist a natural number coding a formal derivation within the system F whose conclusion is a syntactic contradiction." The syntactic contradiction is often taken to be "0=1", in which case Cons(F) states "there is no natural number that codes a derivation of '0=1' from the axioms of F."\\
This theorem is stronger than the first incompleteness theorem because the statement constructed in the first incompleteness theorem does not directly express the consistency of the system.
 \subsection{Relationship with computability}
The inaccuracy theorem is strongly connected with several outcomes in recursion theory
on undecidable sets.Stephen Cole Kleene (1943) provided evidence of G¨odel’s theorem of
incompleteness with basic computer theory results. One result shows that the problem of
stopping is undecidable. No programme can determine correctly if any P programme is used
as an input if P will stop when running with a certain input. Kleene demonstrated that the
existence of a complete, effective arithmetic system with certain coherence characteristics
would lead to a definable contradiction in the stopping problem.



 



 




















\end{document}