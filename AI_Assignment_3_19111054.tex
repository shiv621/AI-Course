
\documentclass[10pt,a4paper,twoside]{article}
\usepackage[dutch]{babel}
\usepackage{amssymb}
\usepackage{amsmath}
\usepackage{float,flafter}	
\usepackage{hyperref}
\usepackage{inputenc}
\setlength\paperwidth{20.999cm}\setlength\paperheight{29.699cm}\setlength\voffset{-1in}\setlength\hoffset{-1in}\setlength\topmargin{1cm}\setlength\headheight{12pt}\setlength\headsep{0cm}\setlength\footskip{1.131cm}\setlength\textheight{25cm}\setlength\oddsidemargin{2.499cm}\setlength\textwidth{15.999cm}

\begin{document}
\begin{center}
\hrule

\vspace{.4cm}
{\bf {\Large ASSIGNMENT-3}}\\
\vspace{.3cm}
{\bf {\huge  Moravec's Paradox }}
\vspace{.3cm}
\end{center}
{\bf Name:}  Shivam Yadav\\
{\bf Roll no:}  19111054 \\
{\bf Branch: }  Biomedical Engineering \hspace{\fill}  28 July, 2021 \\
\hrule

\vspace{.5cm}






\section{Introduction}
Moravec's paradox is the observation by artificial intelligence and robotics researchers that, contrary to traditional assumptions, reasoning requires very little computation, but sensorimotor skills require enormous computational resources. Moravec wrote in 1988, "it is comparatively easy to make computers exhibit adult level performance on intelligence tests or playing checkers, and difficult or impossible to give them the skills of a one-year-old when it comes to perception and mobility".

 Similarly, Minsky emphasized that the most difficult human skills to reverse engineer are those that are unconscious. "In general, we're least aware of what our minds do best", he wrote, and added "we're more aware of simple processes that don't work well than of complex ones that work flawlessly"
\section{The biological basis of human skills  }
All human skills are implemented biologically, using machinery designed by the process of natural selection. The older a skill is, the more time natural selection has had to improve the design. Abstract thought developed only very recently, and consequently, we should not expect its implementation to be particularly efficient.

As Moravec writes:

Encoded in the large, highly evolved sensory and motor portions of the human brain is a billion years of experience about the nature of the world and how to survive in it. it just seems so when we do it.

A compact way to express this argument would be:

We should expect the difficulty of reverse-engineering any human skill to be roughly proportional to the amount of time that skill has been evolving in animals.

The oldest human skills are largely unconscious and so appear to us to be effortless.

Therefore, we should expect skills that appear effortless to be difficult to reverse-engineer, but skills that require effort may not necessarily be difficult to engineer at all.

Some examples of skills that have been evolving for millions of years: recognizing a face, moving around in space, judging people's motivations, catching a ball, recognizing a voice, setting appropriate goals, paying attention to things that are interesting; anything to do with perception, attention, visualization, motor skills, social skills and so on.


\section{Historical influence on Artificial intelligence}

In the early days of artificial intelligence research, leading researchers often predicted that they would be able to create thinking machines in just a few decades (see history of artificial intelligence). Their optimism stemmed in part from the fact that they had been successful at writing programs that used logic, solved algebra and geometry problems and played games like checkers and chess. Many prominent researchers assumed that, having (almost) solved the "hard" problems, the "easy" problems of vision and commonsense reasoning would soon fall into place. The fact that they had solved problems like logic and algebra was irrelevant, because these problems are extremely easy for machines to solve.

Rodney Brooks explains that, according to early AI research, intelligence was "best characterized as the things that highly educated male scientists found challenging", such as chess, symbolic integration, proving mathematical theorems and solving complicated word algebra problems. "The things that children of four or five years could do effortlessly, such as visually distinguishing between a coffee cup and a chair, or walking around on two legs, or finding their way from their bedroom to the living room were not thought of as activities requiring intelligence."


 



 




















\end{document}