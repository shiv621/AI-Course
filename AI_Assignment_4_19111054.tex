
\documentclass[10pt,a4paper,twoside]{article}
\usepackage[dutch]{babel}
\usepackage{amssymb}
\usepackage{amsmath}
\usepackage{float,flafter}	
\usepackage{hyperref}
\usepackage{inputenc}
\setlength\paperwidth{20.999cm}\setlength\paperheight{29.699cm}\setlength\voffset{-1in}\setlength\hoffset{-1in}\setlength\topmargin{1cm}\setlength\headheight{12pt}\setlength\headsep{0cm}\setlength\footskip{1.131cm}\setlength\textheight{25cm}\setlength\oddsidemargin{2.499cm}\setlength\textwidth{15.999cm}

\begin{document}
\begin{center}
\hrule

\vspace{.4cm}
{\bf {\Large ASSIGNMENT-4}}\\
\vspace{.3cm}
{\bf {\huge AI in Architecture  }}
\vspace{.3cm}
\end{center}
{\bf Name:}  Shivam Yadav\\
{\bf Roll no:}  19111054 \\
{\bf Branch: }  Biomedical Engineering \hspace{\fill}  7 August, 2021 \\
\hrule

\vspace{.5cm}






\section{Intoduction}

Architecture, the art and technique of designing and building, as distinguished from the skills associated with construction. The practice of architecture is employed to fulfill both practical and expressive requirements, and thus it serves both utilitarian and aesthetic ends. Because every society—settled or nomadic—has a spatial relationship to the natural world and to other societies, the structures they produce reveal much about their environment (including climate and weather), history, ceremonies, and artistic sensibility, as well as many aspects of daily life.


\section{5 Ways Artificial Intelligence Can Reshape Architecture   }
\subsection{      Parametric Architecture      }

Parametric architecture is a hidden weapon that allows an architect to change specific parameters to create various types of output designs and create such structures that would not have been imagined earlier. A process like this allows Artificial Intelligence to reduce the effort of an architect so that the architect can freely think about different ideas and create something new.
\subsection{  Construction and Planning }

Constructing a building is not a one-day task as it needs a lot of pre-planning. However, this pre-planning is not enough sometimes, and you need a little bit of more effort to get an architect’s opinion to life. Artificial Intelligence will make an architect’s work significantly easier by analyzing the whole data and creating models that can save a lot of time and energy of the architect.

\subsection{ Laying the Foundation }

Constructing a building is not a one-day task as it needs a lot of pre-planning. Artificial Intelligence will make an architect’s work significantly easier by analyzing the whole data and creating models that can save a lot of time and energy of the architect.

All in all, AI can be called an estimation tool for various aspects while constructing a building.
\subsection{  AI for Making Homes }
This era of modernization demands everything to be smartly designed. Just like smart cities, today’s high technology society demands smart homes. However, now architects do not have to bother about how to use AI to create the designs of home only, but they should worry about making the user’s experience worth paying.
\subsection{   Smart Cities }
Change is something that should never change. The way your city looks today will be very different in the coming time. The most challenging task for an architect is city planning that needs a lot of precision planning. However, the primary task is to analyze all the possible aspects, and understand how a city will flow, how the population is going to be in the coming years.
 




















\end{document}